\documentclass[hidelinks]{article}

%% Deutsche Silbentrennung und Sprache (neue Rechtschreibung)
\usepackage[ngerman]{babel}
%% Verwende Umlaute direkt
\usepackage[utf8x]{inputenc}
%% Hyperlinks für interne Referenzen
\usepackage{hyperref}
%% Grafiken einbinden
\usepackage{graphicx}
\usepackage{float}
%% Paket für Unterabbildungen pro Abbildung
\usepackage{subfig}


\usepackage{enumerate}
\usepackage{amsmath}
\usepackage{amssymb}

\usepackage{mathtools}
% Proof system
\usepackage{amsthm}

\theoremstyle{plain}
\newtheorem{thm}{Satz}[section]
\newtheorem{lem}[thm]{Lemma}
\newtheorem{bsp}[thm]{Beispiel}

\theoremstyle{definition}
\newtheorem{defn}[thm]{Definition}

\theoremstyle{remark}
\newtheorem*{remark}{Bemerkung}
%\usepackage{xpatch}
%\makeatletter
%% Remove last point from definitions, theorems, etc.
%\xpatchcmd{\@thm}{\thm@headpunct{.}}{\thm@headpunct{\\}}{}{}
%\makeatother


% Seitenränder
\usepackage[margin=1.5in]{geometry}
% Zitate
\usepackage{cite}
% Tabellen
\usepackage[table]{xcolor}
\definecolor{lightblue}{HTML}{8ADAF2}
\definecolor{lightred}{HTML}{F2A28A}
\definecolor{lightgreen}{HTML}{A6F28A}

% Graphs
\usepackage{tikz}
\usepackage{pgfplots}
\usepgfplotslibrary{fillbetween}
\pgfplotsset{every axis/.append style={
                    axis x line=middle,    % put the x axis in the middle
                    axis y line=middle,    % put the y axis in the middle
                    axis line style={->,color=black}, % arrows on the axis
                    xlabel={$x$},          % default put x on x-axis
                    ylabel={$y$},          % default put y on y-axis
				}}

\setlength{\parindent}{0pt}

\newcommand{\pgtwo}{PG(2, $\mathbb{F}$)\ }
\newcommand{\fnz}{\mathbb{F}\setminus\{0_K\}}
\newcommand{\ftnz}{\mathbb{F}^{3}\setminus\{0_V\}}
\newcommand{\pu}{\mathcal{P}_U}
\newcommand{\gu}{\mathcal{G}_U}

%Rename of Literatur to Literaturverzeichnis
\addto{\captionsngerman}{\renewcommand{\refname}{Literaturverzeichnis}}

% Titel der Arbeit
\title{Elliptische-Kurven-Kryptographie}
% Angaben zum Author
\author{Kevin Kappelmann, Lukas Stevens}
\pagestyle{plain}

%------------------------------------------------------------------------------
\begin{document}

\pagenumbering{gobble}
\maketitle
\newpage
\tableofcontents % Inhaltsverzeichnis
\listoffigures % Abbildungsverzeichnis
\listoftables % Tabellenverzeichnis 
\newpage

\pagenumbering{arabic}

\begin{sloppypar}

\section{Einleitung und Motivation}
Kryptosysteme wie RSA, Diffie-Hellman\footnote{In der jeweiligen Implementierung als Gruppe über ganze Zahlen} und ElGamal\footnotemark[\value{footnote}], die sich auf die Schwere der Primafaktorzerlegung bzw.\ dem diskreten Logarithmenproblem über Ganzzahlen stützen, benötigen sehr große Schlüssellängen, um eine ausreichend hohe Sicherheit zu garantieren. 
Daraus ergibt sich sowohl eine hoher Energie- als auch Speicherbedarf für die Berechnung der Algorithmen, was vor allem für Microchips und eingebettete Systeme ein Problem darstellt.\\
Eine Lösung für dieses Problem sind elliptische Kurven. Diese algebraischen Kurven tragen eine Gruppenstruktur, über die das diskrete Logarithmenproblem deutlich schwerer lösbar ist, als über Gruppen mit Ganzzahlen.
Kryptosysteme, die auf elliptische Kurven beruhen, kommen dadurch mit erheblich kürzeren Schlüsseln bei vergleichbarer Sicherheit aus.\cite[Seite~53]{nist}\\
Nachfolgende Tabelle verdeutlicht diesen Sachverhalt. Spalte 1 kennzeichnet die maximale Sicherheit (in Bits) für den jeweiligen Algorithmus und der angegebenen Schlüssellänge (in Bits). Rot markierte Felder gelten als kryptographisch unsicher, grüne als aktuell sicher.
\begin{table}[h]
\centering
	\begin{tabular}{| c | c | c |}
	\hline
	\rowcolor{lightblue}
	Sicherheitsniveau & RSA/Diffie-Hellman\footnotemark[\value{footnote}] & Elliptische-Kurven\\ \hline
	\rowcolor{lightred}
	$\le80$ 	& 1024 & 160-223 \\ \hline
	\rowcolor{lightgreen}
	112 	& 2048 & 224-255 \\ \hline
	\rowcolor{lightgreen}
	128 	& 3072 & 256-383 \\ \hline
	\rowcolor{lightgreen}
	192 	& 7680 & 384-511 \\ \hline
	\rowcolor{lightgreen}
	256 	& 15360 & 512+ \\ \hline
	\end{tabular}
\caption{Vergleich Schlüssellängen}
\end{table}

Die Verwendung elliptischer Kurven in der Kryptographie wurde Mitte der 1980er Jahre von Neal Koblitz\cite{koblitz} und Victor S. Miller\cite{miller} unabhängig voneinander vorgeschlagen. Aufgrund der vorteilhaften Eigenschaften gewinnt die \textbf{Elliptische-Kurven-Kryptographie} (kurz \textbf{ECC} für Elliptic Curves Cryptography) stets mehr an Bedeutung und löst ältere Verfahren wie RSA in den verschiedensten Bereichen ab. Vor allem in Umgebungen mit begrenzten Kapazitäten, wie z.B.\ Smartcards, ist ECC bereits weit verbreitet.\\
So verwendet beispielsweise Österreich seit 2004 als Vorreiter für alle gängigen Bürgerkarten ECC.\cite{austria} Aber auch die Reisepässe der meisten Europäischen Staaten nutzen inzwischen meist in einer Form ECC.\cite{eu}


\section{Grundbegriffe}
Um elliptische Kurven einführen zu können, müssen wir uns zunächst mit affiner und projektiver Geometrie und ihrer Verwandtheit auseinander setzen. Wir führen hierfür zunächst allgemein die Begriffe der affinen und projektiven Ebene ein und konstruieren uns eine projektive Ebene \pgtwo über einen beliebigen Körper $(\mathbb{F},+,*)$.\\\\
In den folgenden Kapiteln kürzen wir zu Gunsten der Notation den Körper $(\mathbb{F},+,*)$ mit $\mathbb{F}$ ab.\\


\subsection{Affine Ebenen}
\begin{defn} Es sei $\mathcal{A}$ eine Menge und $\mathcal{G}$ eine Teilmenge der Potenzmenge von $\mathcal{A}$, d.h. $\mathcal{G}\subseteq Pot(\mathcal{A})$.
Die Menge $\mathcal{A}$ nennt man die \textbf{Punktmenge} und die Menge $\mathcal{G}$ die \textbf{Geradenmenge} der affinen Ebene $(\mathcal{A},\mathcal{G})$, falls folgende drei Bedingungen erfüllt sind:
	\begin{enumerate}
		\item[(A1)] Zu je zwei Elementen $a, b\in \mathcal{A}$ mit $a\ne b$ existiert genau ein $G\in\mathcal{G}$ mit $a, b \in G$ (durch zwei verschiedene Punkte geht genau eine Gerade).\\
		Wir schreiben $\overline{a,b}$ für dieses G.
		\item[(A2)] Zu $G\in\mathcal{G}$ und $a\in\mathcal{A}\setminus G$ existiert genau ein $G'\in\mathcal{G}$ mit $a\in G'$ und $G\cap G'=\emptyset$ (durch jeden Punkt geht genau eine Gerade, die zu einer gegebenen Gerade parallel ist).\\
		Das sogenannte \textbf{Parallelenaxiom}.
		\item[(A3)] Es existieren drei Elemente $a,b,c\in\mathcal{A}$ mit $c\notin\overline{a,b}$ (es gibt drei Punkte, die nicht alle auf einer Gerade liegen).
	\end{enumerate}
\end{defn}

\begin{bsp}
Das \textbf{Minimalmodell} einer affinen Ebene umfasst genau 4 Punkte.\cite[Seite~16]{henn}
\begin{figure}[h]
\centering
\begin{tikzpicture}[scale=1.1,node distance=2.5cm,main node/.style={circle,fill=black,scale=0.75}]

\node at (0,0) [main node] (A) [label={[label]180:$a$}] {}; 
\node at (1.25,-2.5) [main node] (B) [label={[label]180:$b$}] {}; 
\node at (2.5,0)[main node] (D) [label={[label]0:$c$}] {};
\node at (1.25,-0.8) [main node] (C) [label={[label]0:$d$}] {};

\node at (3,-1.25) {$\boldsymbol\cong$};

\draw (A) -- (B) -- (C) -- (D) -- (A) -- (C)
      (D) -- (B);

\end{tikzpicture}
\begin{tikzpicture}[scale=1.1,node distance=2.5cm,main node/.style={circle,fill=black,scale=0.75}]

\node at (0,0) [main node] (A) [label={[label]180:$a$}] {}; 
\node at (0,-2.5) [main node] (B) [label={[label]180:$b$}] {}; 
\node at (2.5,-2.5)[main node] (C) [label={[label]0:$c$}] {};
\node at (2.5,0) [main node] (D) [label={[label]0:$d$}] {};

\draw (A) -- (B) -- (C) -- (D) -- (A) -- (C)
      (D) -- (B);
\end{tikzpicture}
\caption{Minimalmodell einer affinen Ebene}
\end{figure}

\end{bsp}

\begin{bsp}
Die euklidische Ebene (``Der zweidimensionale Raum unserer Anschauung'') ist eine affine Ebene, in der zusätzich Längen- und Winkelmaß definiert sind.
\begin{figure}[h]
\centering
 \begin{tikzpicture}[scale=0.9,domain=-3:3] 
    \draw[step=1cm,very thin,color=gray] (-2.9,-2.9) grid (2.9,2.9);
    \draw[thick, ->] (-3.2,0) -- (3.2,0) node[right] {$x$}; 
    \draw[thick, ->] (0,-3.2) -- (0,3.2) node[above] {$y$};
    \foreach \x/\xtext in {-2, -1, 1, 2} 
	\draw (\x cm,1pt) -- (\x cm,-1pt) node[anchor=north,fill=white] {$\xtext$};
    \foreach \y/\ytext in {-2, -1, 1, 2} 
	\draw (1pt,\y cm) -- (-1pt,\y cm) node[anchor=east,fill=white] {$\ytext$};
    \draw[red] (-1.5,-0.5) node[right] {$a$};
    \draw[red] (1.5,2.5) node[right] {$b$};
    \draw[blue] (-1.5,-2.5) node[right] {$c$};
    \draw[blue] (1.5,0.5) node[right] {$d$};
    \draw[domain=-3:2,color=red]    plot (\x,1+\x)   node[right] {$\overline{a,b}$}; 
    \draw[domain=-2:3,color=blue]    plot (\x,-1+\x)   node[right] {$\overline{c,d}$};
    \filldraw[red] (-1.5,-0.5) circle[radius=2pt];
    \filldraw[red] (1.5,2.5) circle[radius=2pt];
    \filldraw[blue] (-1.5,-2.5) circle[radius=2pt];
    \filldraw[blue] (1.5,0.5) circle[radius=2pt];
    \draw (4,3) node[draw] {$\overline{a,b}\parallel\overline{c,d}$};
   \end{tikzpicture}
 \caption{Parallelen in der euklidischen Ebene}
\end{figure}
\end{bsp}

\subsection{Projektive Ebenen}
\begin{defn} ps sei $\mathcal{P}$ eine Menge und $\mathcal{G}$ eine Teilmenge der Potenzmenge von $\mathcal{P}$, d.h. $\mathcal{G}\subseteq Pot(\mathcal{P})$.
Die Menge $\mathcal{P}$ nennt man die \textbf{Punktmenge} und die Menge $\mathcal{G}$ die \textbf{Geradenmenge} der projektiven Ebene $(\mathcal{P},\mathcal{G})$, falls folgende drei Bedingungen erfüllt sind:
	\begin{enumerate}
		\item[(P1)] Zu je zwei Elementen $P, Q\in \mathcal{A}$ mit $P\ne Q$ existiert genau ein $G\in\mathcal{G}$ mit $P, Q \in G$ (durch zwei verschiedene Punkte geht genau eine Gerade).\\
		Wir schreiben $\overline{P,Q}$ für dieses G.
		\item[(P2)] Für je zwei $G,H\in\mathcal{G}$ mit $G\ne H$ gilt $|G\cap H|=1$ (zwei verschiedene Geraden schneiden sich in genau einem Punkt).
		\item[(P3)] Es existieren vier verschiedene Elemente in $\mathcal{P}$, sodass immer höchstens zwei davon in jedem beliebigen $G\in\mathcal{G}$ liegen (es gibt vier Punkte, sodass nie drei davon auf derselben Gerade liegen).
	\end{enumerate}
\end{defn}
Im wesentlichen Unterschied zu affinen Ebenen existieren in einer projektiven Ebene \textbf{keine Parallelen}.

\begin{bsp}
Die \textbf{Fano-Ebene} ist das Minimalmodell einer projektiven Ebene und umfasst genau 7 Punkte (beachte: auch der Kreis gilt hier als Gerade!).\cite[Seite~9]{projmin}\\
Bemerkenswert ist die Tatsache, dass durch Entfernen einer beliebigen Gerade und den daraufliegenden Punkten eine affine Ebene entsteht. Dies ist kein Spezialfall sondern funktioniert immer, was wir auch im Abschnitt~\ref{konstr-aff-aus-proj} zeigen werden.
\begin{figure}[h]
\centering
\begin{tikzpicture}[node distance=1.5cm,main node/.style={circle,fill=black,scale=0.75}]

\node at (1.5,0) [main node] (A) [label={[label]180:$A$}] {}; 
\node at (0.63,-1.5) [main node] (B) [label={[label]180:$B$}] {}; 
\node at (-0.15,-3)[main node] (C) [label={[label]270:$C$}] {};
\node at (1.5,-3) [main node] (D) [label={[label]270:$D$}] {};
\node at (3.15,-3) [main node] (E) [label={[label]270:$E$}] {};
\node at (2.37,-1.5) [main node] (F) [label={[label]0:$F$}] {};
\node at (1.5,-2) [main node] (G) [label={[label]0:$G$}] {};

\draw (A) -- (B) -- (C) -- (D) -- (E) -- (F) -- (A)
	(B) -- (G) -- (D)
    (G) -- (F);
\draw (G) -- (E);
\draw (G) -- (A);
\draw (G) -- (C);

\draw (1.5,-2) circle [radius=1];

\end{tikzpicture}
\caption{Fano-Ebene}
\end{figure}
\end{bsp}


\subsubsection{Die projektive Ebene \pgtwo}
Es sei $\mathbb{F}$ ein beliebiger Körper mit Nullelement $0_K$ und $\mathbb{F}^3$ der dreidimensionale $\mathbb{F}$-Vektorraum mit Nullvektor $0_V$. Wir definieren eine Äquivalenzrelation~$\sim$ für alle $a,b\in\ftnz$ wie folgt:\\
\begin{equation*}
	a\sim b :\Leftrightarrow \exists\lambda\in\fnz:\lambda a=b
\end{equation*}
Für die Äquivalenzklassen von $a=(a_1,a_2,a_3)\in\ftnz$ schreiben wir $[a]$ oder auch $(a_1:a_2:a_3)$.\\

Weiters definieren wir die Menge aller Äquivalenzklassen als unsere Punktemenge:
\begin{equation*}
	\mathcal{P}:=\Bigl \{[a]\mid a\in\ftnz\Bigr \}
\end{equation*}
Für zwei verschiedene Punkte $P=[a]\in\mathcal{P}$ und $Q=[b]\in\mathcal{P}$ setzen wir die Verbindungsgerade zwischen $P$ und $Q$ fest mit:
\begin{equation*}
	\overline{P,Q}:=\Bigl \{[\lambda a+\mu b]\mid\lambda,\mu\in\fnz\Bigr\}
\end{equation*}
Womit wir nun auch die Menge aller Geraden bilden können:
\begin{equation*}
	\mathcal{G}:=\{\overline{P,Q}\mid P,Q\in\mathcal{P}\wedge P\ne Q\}
\end{equation*}

\subsubsection{Konstruktion affiner Ebenen aus projektiven Ebenen} \label{konstr-aff-aus-proj}

Beweis, Beispiel
\begin{lem} \label{projektive-zu-affinen}
    $\pu$ und $\gu$ definieren.
\end{lem}
\section{Elliptische Kurven $E$}
%Macht Lukas
\subsection{Die unendliche Gerade über \pgtwo}
%Isomorphismus von $\mathbb{F}^2 \rightarrow \mathcal{P}_U$
Um in~\ref{definition-ek} elliptische Kurven genau beschreiben zu können und in~\ref{affine-darstellung} eine affine Darstellung elliptischer Kurven herzuleiten, müssen wir $(\mathcal{P,G}) =$ \pgtwo nochmal betrachten.
Wir wählen dazu eine Gerade $U \in \mathcal{G}$ aus. 
Prinzipiell kann dazu jede Gerade gewählt werden. 
Es ist jedoch von Vorteil eine bestimmte Gerade zu wählen um das Rechnen mit der Weierstraßgleichung(\ref{weierstrass}) zu vereinfachen. \\
\newline
Dazu wählen wir die Verbindungsgerade $U = \overline{P,Q}$ der Punkte $P = (1:0:0)$ und $Q = (0:1:0)$, d.h.\ $U = \left\{ (x:y:z) \in \mathcal{P} \mid z = 0 \right\}$. 
Diese Menge $U$ bezeichnen wir im Folgenden als unendlich ferne Gerade.
Im dreidimensionalen Raum ist das genau die $x,y$-Ebene mit $z=0$. \\ 
\begin{lem}[Isomorphismus von $\pu$ und $\mathbb{F}^2$]
    Gegeben die projektive Ebene $(\mathcal{P}, \mathcal{G}) = PG(2,\mathbb{F})$ und die unendlich ferne Gerade $U$, dann ist die Abbildung
    \begin{equation*}
        \phi: \mathbb{F}^2 \rightarrow \mathcal{P}_U,\ (a,b) \mapsto (a:b:1)
    \end{equation*}
    bijektiv und bildet Geraden auf Geraden ab, d.h. $\phi$ ist ein Isomorphismus von affinen Ebenen. 
\end{lem}
\begin{proof} \label{isomorphismus-lemma}
    Wie im Lemma~\ref{projektive-zu-affinen} gezeigt wurde, erhält man eine affine Ebene, wenn man aus einer projektiven Ebene eine Gerade mitsamt allen ihren Punkten entfernt. Daraus folgt, dass es sich bei $(\mathcal{P}_U, \mathcal{G}_U)$ um eine affine Ebene handelt. Es sei $(a:b:c) \in \mathcal{P}_U$. Da gilt $(a:b:c) \notin U$, folgt $c \neq 0$. Das heißt $c^{-1}$ ist definiert, womit die Abbildung
    \begin{equation*}
        \phi(ac^{-1},bc^{-1}) = (ac^{-1}:bc^{-1}:1) = (a:b:c)
    \end{equation*}
    surjektiv ist. Die Injektivität gilt auch, da mit $(a,b) \neq (a',b')$ die Vektoren $(a,b,1)$ und $(a',b',1)$ linear unabhängig sind, womit $(a:b:1) \neq (a':b':1)$ folgt. \\
    \newline
    Jede Gerade in $\mathbb{F}^2$ ist von der Form $\overline{a,b} = \left\{ a + \lambda b \mid \lambda \in \mathbb{F}, a,b \in \mathbb{F}^2, b \neq (0,0) \right\}$. Für einen Punkt $P = a + \lambda b \in \overline{a,b}$ gilt dann:
    \begin{equation*}
        \begin{aligned}
            \phi(a + \lambda b) & = (a_1 + \lambda b_1 : a_2 + \lambda b_2 : 1) = (a_1 : a_2 : 1) + \lambda (b_1 : b_2 : 0) \\
            & \sim \mu (a_1 : a_2 : 1) + \mu \lambda (b_1 : b_2 : 0)
        \end{aligned}
    \end{equation*}
    Hierbei ist zu beachten, dass $\mu \in \mathbb{F} \setminus \{0\}$ laut Definition der Äquivalenzrelation $\sim$ gilt. Man betrachte nun die Gerade
    \begin{equation*}
        G \coloneqq \left\{ u(a_1 : a_2 : 1) + v(b_1 : b_2 : 0) \mid (u,v) \in \mathbb{F}^2  \setminus \{(0,0)\} \right\}.
    \end{equation*}
    Alle Punkte der Bildmenge von $\phi$ liegen auf der Gerade $G$. Es wird nur ein Punkt nicht erreicht, nämlich der Punkt $R = (b_1 : b_2 : 0)$. Wie man sehen kann, gilt $G \cap U = R$. Es folgt $\phi(\overline{a,b}) = G \cap \pu \in \gu$.
\end{proof}
Insgesamt kann man sehen, dass man affine Geraden auf eine Teilmenge der projektiven Geraden abbilden kann. Außerdem bekommen diese affinen Geraden im Projektiven dann einen Schnittpunkt, der auf der unendlich fernen Gerade $U$ liegt.
\subsection{Definiton elliptischer Kurven} \label{definition-ek}
%Weierstraßgleichung, Nullstellenmenge des Polynoms, Charakteristiken(Singularitäten), affine Koordinatentransformation?
Wir haben bereits die projektive Ebene \pgtwo über beliebige Körper $\mathbb{F}$ eingeführt.\\
Diese hat die folgende Punktemenge:
\begin{equation*}
    P = \left\{(u:v:w) \mid (u,v,w) \in \mathbb{F}^3 \setminus (0,0,0) \right\}
\end{equation*}
Nun wollen wir die Punktemenge $E$ der elliptischen Kurve einführen, welche eine Teilmenge der Punktemenge $\mathcal{P}$ ist, d.h. $E \subseteq \mathcal{P}$. 
Dazu benötigen wir Polynome in drei Unbekannten.
Der Polynomring mit drei Unbekannten über $\mathbb{F}$ ist mit 
\begin{equation*}
    \mathbb{F}[X,Y,Z] = \left\{ \sum_{k,l,m \geq 0} a_{k,l,m} \thinspace X^k Y^l Z^m \mid a_{k,l,m} \in \mathbb{F} \right\}
\end{equation*}
definiert. 
$F(X,Y,Z) = \sum_{k,l,m \geq 0} a_{k,l,m} \thinspace X^k Y^l Z^m \in \mathbb{F}[X,Y,Z]$ wird Polynom genannt. 
\begin{defn}[Elliptische Kurve] \label{weierstrass}
    Eine elliptische Kurve $E$ ist durch die Lösung der Weierstraß-Gleichung 
    \begin{equation*}
        Y^2Z + a_1XYZ + a_3YZ^2 = X^3 + a_2X^2Z + a_4XZ^2 + a_6Z^3
    \end{equation*}
    gegeben, wobei $a_i \in \mathbb{F}$ gilt und die Lösung keine Singularitäten besitzen darf.\cite[Seite~54]{milne2006}
    Da der zugrundeliegende Raum \pgtwo eine projektive Ebene ist, handelt es sich um eine projektive Kurve. 
    Wenn man die Gleichung als Polynom 
    \begin{equation*}
        F(X,Y,Z) = Y^2Z + a_1XYZ + a_3YZ^2 - X^3 - a_2X^2Z - a_4XZ^2 -a_6Z^3
    \end{equation*}
    schreibt, dann ist $E$ genau die Nullstellenmenge des Polynoms $F$. Bemerkenswert ist hier, dass es sich um ein homogenes Polynom vom Grad 3 handelt, d.h.\ für jedes Summenglied $a_{k,l,m}X^kY^lZ^m$ mit $a_{k,l,m}$ gilt $k + l + m = 3$.
\end{defn}
\begin{defn}[Singularitäten]
    Eine Kurve $E$ ist singulär in einem Punkt $P=(a:b:c)\in E$, wenn gilt
    \begin{equation*}
        \frac{\partial F}{\partial X}(P) = \frac{\partial F}{\partial Y}(P) =  \frac{\partial F}{\partial Z}(P) = 0
    \end{equation*}
    Man sagt auch, dass die partiellen Ableitungen des Polynoms $F$ im Punkt $P$ verschwinden. Falls die elliptische Kurve $E$ in keinem Punkt singulär ist, dann bezeichnet man sie als nicht-singulär.\cite[Seite~227]{karpfinger-kiechle}
\end{defn}
\begin{bsp}[Singularitäten]
    Die folgenden Kurven sind jeweils in einem Punkt singulär. Damit gibt es mehrere Tangenten an diesen Punkt.
    \begin{figure}[H]
        \centering
        \subfloat[$y^2=x^3+x^2$]{
            \begin{tikzpicture}
                \begin{axis}[
                    scale=0.8,
                    ticks = none,
                    axis equal,
                    restrict y to domain = -4:4,
                    restrict x to domain = -1:1.6,
                    ]
                    \addplot [color=red, domain = -4:4, samples = 300, unbounded coords=jump]
                    ({1/(x^3 - 1)}, { (x/(x^3 - 1))^1.5});
                    \addplot [color=red, domain = -4:4, samples = 300, unbounded coords=jump]
                    ({1/(x^3 - 1)}, {-(x/(x^3 - 1))^1.5}); 
                \end{axis}
            \end{tikzpicture}}
        \qquad
        \subfloat[$y^2=x^3$]{
            \begin{tikzpicture}
                \begin{axis}[
                    scale=0.8,
                    ticks = none,
                    axis equal,
                    ]
                    \addplot [color=red, domain=-2:2, samples=300, unbounded coords=jump]
                    {sqrt(x^3)};
                    \addplot [color=red, domain=-2:2, samples=300, unbounded coords=jump]
                    {-sqrt(x^3)};
                \end{axis}
            \end{tikzpicture}}
        \caption{Kurven mit Singularitäten(Knoten und Spitze)}
    \end{figure}
\end{bsp}
Wir hatten eine elliptische Kurve $E$ als Nullstellenmenge des Polynoms $F(X,Y,Z)$ mit $E \coloneqq \left\{(u:v:w) \in \mathcal{P} \mid F(u,v,w) = 0 \right\}$ definiert.
Es handelt sich bei Punkten in der projektiven Ebene und damit auch bei den Elementen von $E$ jedoch um Äquivalenzklassen. Deswegen müssen wir noch die Wohldefiniertheit der Nullstellen begründen. 
Wir rufen uns dazu noch einmal die Definition der Äquivalenzrelation $\sim$ ins Gedächtnis: $(u:v:w) \sim (u':v':w') \iff \exists \lambda \in \mathbb{F} \setminus {0}: (u,v,w) = \lambda (u',v',w')$.
Wir setzen ein:
\begin{equation*}
    F(u',v',w') = F(\lambda u, \lambda v, \lambda w) = \lambda^3 F(u,v,w).
\end{equation*}
Die zweite Äquivalenz folgt aus der Homogenität des Polynoms. Daraus folgt, dass die Nullstellen von $F$ in $\mathcal{P}$ wohldefiniert sind:
\begin{equation*}
    F(u,v,w) = 0 \iff F(\lambda u, \lambda v, \lambda w) = 0.
\end{equation*}
\newline
Wir wollen nun noch eine Einschränkung treffen; die Charakteristik des Körpers $\mathbb{F}$ soll nicht 2 und nicht 3 sein. Wir schreiben $char \: \mathbb{F} \neq 2$ bzw.\ $char \: \mathbb{F} \neq 3$. 
Dies bedeutet, dass $1 + 1 \neq 0$ bzw.\ $1 + 1 + 1 \neq 0$, oder anders gesagt: Wenn wir das neutrale Element der Multiplikation 2 bzw.\ 3 mal addieren, dann erhalten wir nicht das neutrale Element der Addition, welches kein multiplikatives Inverses hat. 
Dadurch wird die Allgemeinheit für den Fall, dass $\mathbb{F}$ eine dieser Charakteristiken hat, eingeschränkt. \\
\newline
%Zitat? Seite 50 Milner
Wir können jetzt die Weierstraßgleichung(\ref{weierstrass}), welche die Form
\begin{equation*}
    Y^2Z + a_1XYZ + a_3YZ^2 = X^3 + a_2X^2Z + a_4XZ^2 + a_6Z^3
\end{equation*}
hat, umformen. Zuerst können wir, wenn $char \: \mathbb{F} \neq 2$ gilt, den Term $XYZ$ mit folgendem Variablenwechsel eliminieren:
\begin{equation*}
    X' = X, \: Y' = Y + \frac{a_1}{2} X, \: Z' = Z 
\end{equation*}
Anschließen könnnen wir auch noch die Terme $X^2$ und $Y$ eliminieren, falls $char \: \mathbb{F} \neq 2,3$ gilt:
\begin{equation*}
    X' = X + \frac{a_2}{3}, \: Y' = Y + \frac{a_3}{2}, \: Z' = Z
\end{equation*}
Damit lautet das Ergebnis
\begin{equation} \label{reduziert-weierstrass}
    Y^2Z = X^3 + aXZ^2 + bZ^3.
\end{equation}
\subsection{Affine Darstellung elliptischer Kurven} \label{affine-darstellung}
Wir wollen eine affine Darstellung herleiten. Dazu zeigen wir zunächst, dass nur ein Punkt der unendlich fernen Gerade $U$, nämlich der Punkt $\mathcal{O} = (0:1:0)$, auf $E$ liegt.
Für $P \in U$ gilt $P = (u:v:0)$ mit $u,v \in \mathbb{F}$. Es gibt also, bis auf Äquivalenz, nur drei verschiedene Punkte: $P = (1:0:0)$, $Q = (1:1:0)$ und $\mathcal{O} = (0:1:0)$. Wenn wir diese Punkte in die Gleichung~\ref{reduziert-weierstrass} einsetzen, dann löst nur $\mathcal{O}$ die Gleichung. \\ 
Für jeden Punkt $P \in E$ mit $P \neq Q$ gilt also $P \in \pu$. 
Aufgrund der Äquivalenzrelation $\sim$ können wir ohne Einschränkung der Allgemeinheit annehmen, dass $P \in \left\{(u:v:1) \mid u,v \in \mathbb{F} \right\}$. Wenn wir also nur diese Punkte betrachten können wir die Gleichung~\ref{reduziert-weierstrass} vereinfachen und erhalten $y^2 = x^3 + ax + b$ oder als Polynom: 
\begin{equation} \label{affines-polynom}
    f(x,y) \coloneqq y^2 - x^3 - ax - b
\end{equation}
Wir wissen aus Lemma~\ref{projektive-zu-affinen}, dass $\pu$ genau die Punktemenge einer affinen Ebene ist. Wenn wir zusätzlich die Abbildung $\phi$ aus Lemma~\ref{isomorphismus-lemma} auf $\pu$ anwenden, dann zerfällt die Punktemenge der elliptischen Kurve $E$ in zwei Teilmengen, einen affinen Teil und den unendlichen Punkt $\mathcal{O}$:
\begin{equation*}
    E = \left\{(u:v:1) \mid (u,v) \in \mathbb{F}^2 \land f(u,v) = 0 \right\} \cup \left\{ \mathcal{O} \right\}
\end{equation*}
Wir können im Anschluss nur den affinen Teil betrachten, wenn wir den Punkt $\mathcal{O}$ nicht außer Acht lassen.
\begin{bsp}
    Skizzen elliptischer Kurven über dem Körper $\mathbb{R}$
    \begin{figure}[H]
        \centering
        \subfloat[$y^2=x^3 + 0.3x + 2$]{
            \begin{tikzpicture}
                \begin{axis}[
                    no markers,
                    samples=800,
                    scale=0.8,
                    ]
                \addplot gnuplot{atanh(x)};
                \end{axis}
            \end{tikzpicture}}
        \qquad
        \subfloat[$y^2=x^3$]{
            \begin{tikzpicture}
                \begin{axis}[
                    scale=0.8,
                    axis equal,
                    ]
                    \addplot [color=red, domain=-2:2, samples=300, unbounded coords=jump]
                    {sqrt(x^3 - 2*x)};
                    \addplot [color=red, domain=-2:2, samples=300, unbounded coords=jump]
                    {-sqrt(x^3 - 2*x)};
                \end{axis}
            \end{tikzpicture}}
    \end{figure}
\end{bsp}
\section{Eine Gruppe über $E$}
Macht Kevin bis 4.3\\
\subsection{Tangenten elliptischer Kurven}
\subsection{Schnittpunkte von Geraden mit elliptischen Kurven}
Unendlich ferne Gerade mit Schnittpunkt $\mathcal{O}$, Affine Geraden, Parallele zur y-Achse
\subsection{Die Schnittpunkt-Verknüpfung $\oplus $ über $E$}
Definition, Beweis der Abgeschlossenheit, graphische Interpretation
\subsection{Die Gruppe $(E, +)$}
Macht Lukas bis fertig\\
Gruppe ist abelsch mit neutralem Element $\mathcal{O}$, Beispiel
\section{Anwendung elliptischer Kurven in der Kryptologie}
\subsection{ElGamal}
Welche Charakteristiken für elliptische Kurven, Domänenparameter
\subsection{Noch einen für Signaturen}
Welche Charakteristiken für elliptische Kurven, Domänenparameter

\end{sloppypar}
\newpage
\bibliographystyle{plain}
\bibliography{quellen}
\end{document}
