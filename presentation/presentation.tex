%l%%%%%%%%%%%%%%%%%%%%%%%%%%%%%%%%%%%%%%%%
% Beamer Presentation
% LaTeX Template
% Version 1.0 (10/11/12)
%
% This template has been downloaded from:
% http://www.LaTeXTemplates.com
%
% License:
% CC BY-NC-SA 3.0 (http://creativecommons.org/licenses/by-nc-sa/3.0/)
%
%%%%%%%%%%%%%%%%%%%%%%%%%%%%%%%%%%%%%%%%%

%----------------------------------------------------------------------------------------
%	PACKAGES AND THEMES
%----------------------------------------------------------------------------------------

\documentclass{beamer}

\mode<presentation>{% The Beamer class comes with a number of default slide themes
% which change the colors and layouts of slides. Below this is a list
% of all the themes, uncomment each in turn to see what they look like.

%\usetheme{default}
%\usetheme{AnnArbor}
%\usetheme{Antibes}
%\usetheme{Bergen}
%\usetheme{Berkeley}
\usetheme{Berlin}
%\usetheme{Boadilla}
%\usetheme{CambridgeUS}
%\usetheme{Copenhagen}
%\usetheme{Darmstadt}
%\usetheme{Dresden}
%\usetheme{Frankfurt}
%\usetheme{Goettingen}
%\usetheme{Hannover}
%\usetheme{Ilmenau}
%\usetheme{JuanLesPins}
%\usetheme{Luebeck}
%\usetheme{Madrid}
%\usetheme{Malmoe}
%\usetheme{Marburg}
%\usetheme{Montpellier}
%\usetheme{PaloAlto}
%\usetheme{Pittsburgh}
%\usetheme{Rochester}
%\usetheme{Singapore}
%\usetheme{Szeged}
%\usetheme{Warsaw}

% As well as themes, the Beamer class has a number of color themes
% for any slide theme. Uncomment each of these in turn to see how it
% changes the colors of your current slide theme.

%\usecolortheme{albatross}
%\usecolortheme{beaver}
%\usecolortheme{beetle}
%\usecolortheme{crane}
%\usecolortheme{dolphin}
%\usecolortheme{dove}
%\usecolortheme{fly}
%\usecolortheme{lily}
%\usecolortheme{orchid}
%\usecolortheme{rose}
%\usecolortheme{seagull}
%\usecolortheme{seahorse}
%\usecolortheme{whale}
%\usecolortheme{wolverine}

\setbeamertemplate{footline} % To remove the footer line in all slides uncomment this line
%\setbeamertemplate{footline}[page number] % To replace the footer line in all slides with a simple slide count uncomment this line

%\setbeamertemplate{navigation symbols}{} % To remove the navigation symbols from the bottom of all slides uncomment this line
}

%% Deutsche Silbentrennung und Sprache (neue Rechtschreibung)
\usepackage[ngerman]{babel}
%% Verwende Umlaute direkt
\usepackage[utf8x]{inputenc}

\usepackage{graphicx} % Allows including images
\usepackage{booktabs} % Allows the use of \toprule, \midrule and \bottomrule in tables

%% Hyperlinks für interne Referenzen
\usepackage{hyperref}
%% Grafiken einbinden
\usepackage{graphicx}
\usepackage{float}
%% Paket für Unterabbildungen pro Abbildung
\usepackage{subfig}

\usepackage{forloop}

\usepackage{amsmath}
\usepackage{amssymb}
\usepackage{algorithm}
\usepackage[noend]{algpseudocode}
\newcommand*\Let[2]{\State #1 $\gets$ #2}
\algrenewcomment[1]{\(\qquad \triangleright\) #1}
\algrenewcommand\algorithmicrequire{\textbf{Precondition:}}

\usepackage{mathtools}
% Proof system
\usepackage{amsthm}

\theoremstyle{plain}
\newtheorem{thm}{Satz}[section]
\newtheorem{lem}[thm]{Lemma}

\theoremstyle{definition}
\newtheorem{defn}[thm]{Definition}
\newtheorem{bsp}[thm]{Beispiel}

\newtheoremstyle{rem} % name
    {\topsep}                    % Space above
    {\topsep}                    % Space below
    {}                   % Body font
    {}                           % Indent amount
    {\bf}                   % Theorem head font
    {:}                          % Punctuation after theorem head
    {.5em}                       % Space after theorem head
    {}  % Theorem head spec (can be left empty, meaning ‘normal’)

\theoremstyle{rem}
\newtheorem*{remark}{Bemerkung}

%\usepackage{xpatch}
%\makeatletter
%% Remove last point from definitions, theorems, etc.
%\xpatchcmd{\@thm}{\thm@headpunct{.}}{\thm@headpunct{\\}}{}{}
%\makeatother

% Zitate
\usepackage{cite}
% Tabellen
\definecolor{lightblue}{HTML}{8ADAF2}
\definecolor{lightred}{HTML}{F2A28A}
\definecolor{lightgreen}{HTML}{A6F28A}
\definecolor{darkgreen}{HTML}{43A822}
\definecolor{grey}{HTML}{AAAAAA}

% Graphs
\usepackage{tikz}
\usepackage{tikz-3dplot}
\usepackage{pgfplots}
\usepgfplotslibrary{fillbetween}
\pgfplotsset{compat=1.12,
    every axis/.append style={
                    axis x line=middle,    % put the x axis in the middle
                    axis y line=middle,    % put the y axis in the middle
                    axis line style={->,color=black}, % arrows on the axis
                    xlabel={$x$},          % default put x on x-axis
                    ylabel={$y$},          % default put y on y-axis
				}}
\newcommand{\pgtwo}{PG(2, $\mathbb{F}$)\ }
\newcommand{\fnz}{\mathbb{F}\setminus\{0\}}
\newcommand{\ftwnz}{\mathbb{F}^{2}\setminus\{\boldsymbol 0\}}
\newcommand{\ftnz}{\mathbb{F}^{3}\setminus\{\boldsymbol 0\}}
\newcommand{\pu}{\mathcal{P}_U}
\newcommand{\gu}{\mathcal{G}_U}
\newcommand{\patinf}{\mathcal{O}}

%Rename of Literatur to Literaturverzeichnis
\addto{\captionsngerman}{\renewcommand{\refname}{Literaturverzeichnis}}

%----------------------------------------------------------------------------------------
%	TITLE PAGE
%----------------------------------------------------------------------------------------

\title[Elliptische-Kurven-Kryptographie]{Elliptische Kurven Kryptographie} % The short title appears at the bottom of every slide, the full title is only on the title page

\author{Kevin Kappelmann, Lukas Stevens} % Your name
\institute[TUM] % Your institution as it will appear on the bottom of every slide, may be shorthand to save space
{Technische Universität München \\ % Your institution for the title page
}
\date{\today} % Date, can be changed to a custom date

\begin{document}

\begin{frame}
\titlepage % Print the title page as the first slide
\end{frame}

%------------------------------------------------


% Overview slide
%------------------------------------------------

\begin{frame}
	\frametitle{Überblick} % Table of contents slide, comment this block out to remove it
\tableofcontents % Throughout your presentation, if you choose to use \section{} and \subsection{} commands, these will automatically be printed on this slide as an overview of your presentation
\end{frame}

%----------------------------------------------------------------------------------------
%	PRESENTATION SLIDES
%----------------------------------------------------------------------------------------
\section{Elliptische Kurven}
%------------------------------------------------
\subsection{Die unendlich ferne Gerade}
%------------------------------------------------

\begin{frame}
\frametitle{\insertsection~--~\insertsubsection}
\begin{itemize}[<+->]
	\item Wähle $U\coloneqq\overline{P,Q}$ mit $P=(1:0:0),Q=(0:1:0)$.
	\item U ist im dreidimensionalen Raum genau die x,y-Ebene mit $z=0$.

\begin{figure}[H]
\tdplotsetmaincoords{80}{100}
\begin{tikzpicture}[
		tdplot_main_coords,
		axis/.style={->,black},
		line/.style={thick,red},
		scale=0.6]

	% coordinates in (z,x,y)
	\filldraw[
		draw=darkgreen,%
		fill=darkgreen!20,%
	    ]          (0,-2.5,-2.5)
		    -- (0,-2.5,2.5)
		    -- (0,2.5,2.5)
		    -- (0,2.5,-2.5)
		    -- cycle;

	%draw the axes
	\draw[axis] (0,-3,0) -- (0,3,0) node[anchor=west]{$x$};
	\draw[axis] (0,0,-3) -- (0,0,3) node[anchor=west]{$y$};
	\draw[axis] (10,0,0) -- (-10,0,0) node[anchor=west]{$z$};

	\draw[line] (0,-2,0) -- (0,2,0) node[anchor=north]{$Q=(1:0:0)$};
	\draw[line,blue] (0,0,-2) -- (0,0,2) node[anchor=east]{$P=(0:1:0)$};
	\node[darkgreen] at (11,5.9,0) {$U=\overline{P,Q}$};
\end{tikzpicture}
\end{figure}
	\item Wir bezeichnen $U$ als die \textbf{unendlich ferne Gerade}.
\end{itemize}
\end{frame}

%------------------------------------------------

\begin{frame}
\frametitle{\insertsection~--~\insertsubsection}
\begin{lem}
    Gegeben sei die projektive Ebene $(\mathcal{P}, \mathcal{G}) = PG(2,\mathbb{F})$ und die unendlich ferne Gerade $U$, dann ist die Abbildung
    \begin{equation*}
        \phi: \mathbb{F}^2 \rightarrow \mathcal{P}_U,\ (a,b) \mapsto (a:b:1)
    \end{equation*}
    bijektiv und bildet Geraden auf Geraden ab, d.h. $\phi$ ist ein Isomorphismus von affinen Ebenen. 
\end{lem}
\end{frame}

%------------------------------------------------
\subsection{Weierstraß-Gleichung}
%------------------------------------------------

\begin{frame}
\frametitle{\insertsection~--~\insertsubsection}
Erinnerung: Punktemenge von \pgtwo
\begin{equation*}
    P = \left\{(x:y:z) \mid (x,y,z) \in \ftnz \right\}
\end{equation*}
\begin{defn}
Eine elliptische Kurve $E\subseteq P$ ist durch die Lösung der \textbf{Weierstraß-Gleichung} 
    \begin{equation*}
        0=Y^2Z + a_1XYZ + a_3YZ^2 - X^3 - a_2X^2Z - a_4XZ^2 - a_6Z^3
    \end{equation*}
    gegeben, wobei $a_i \in \mathbb{F}$ gilt und die Lösung keine Singularitäten besitzen darf.
\end{defn}
\end{frame}

%------------------------------------------------

\begin{frame}
\frametitle{\insertsection~--~\insertsubsection}
\begin{defn} 
	Eine Kurve $E$ ist \textbf{singulär} in einem Punkt \mbox{$P=(a:b:c)\in E$}, wenn gilt:
    \begin{equation*}
        \frac{\partial F}{\partial X}(P) = \frac{\partial F}{\partial Y}(P) =  \frac{\partial F}{\partial Z}(P) = 0
    \end{equation*}
\end{defn}
    \begin{figure}[H]
        \centering
        \subfloat[$y^2=x^3+x^2$]{
            \begin{tikzpicture}
                \begin{axis}[
                    scale=0.4,
                    ticks = none,
                    axis equal,
                    restrict y to domain = -4:4,
                    restrict x to domain = -1:1.6,
                    ]
                    \addplot [color=red, domain = -4:4, samples = 300, unbounded coords=jump]
                    ({1/(x^3 - 1)}, { (x/(x^3 - 1))^1.5});
                    \addplot [color=red, domain = -4:4, samples = 300, unbounded coords=jump]
                    ({1/(x^3 - 1)}, {-(x/(x^3 - 1))^1.5}); 
                \end{axis}
            \end{tikzpicture}}
        \qquad
        \subfloat[$y^2=x^3$]{
            \begin{tikzpicture}
                \begin{axis}[
                    scale=0.4,
                    ticks = none,
                    axis equal,
                    ]
                    \addplot [color=red, domain=-2:2, samples=300, unbounded coords=jump]
                    {sqrt(x^3)};
                    \addplot [color=red, domain=-2:2, samples=300, unbounded coords=jump]
                    {-sqrt(x^3)};
                \end{axis}
            \end{tikzpicture}}
        \caption{Kurven mit Singularitäten (Knoten und Spitze)}
    \end{figure}
\end{frame}

%------------------------------------------------

\begin{frame}
\frametitle{\insertsection~--~\insertsubsection}
\begin{itemize}[<+->]
	\item  Wir schränken ein: Die \textit{Charakteristik} des Körpers $\mathbb{F}$ soll nicht 2 und nicht 3 sein: $char \: \mathbb{F} \neq 2,3$.
	\item Dies bedeutet, dass $1 + 1 \neq 0$ bzw.\ $1 + 1 + 1 \neq 0$, wobei $0,1$ die neutralen Elemente der Addition bzw.\ Multiplikation von $\mathbb{F}$ sind. 
	\item Unter diesen Voraussetzungen können wir die Weierstraß-Gleichung vereinfachen zu:
    \begin{equation*}
        0=Y^2Z - X^3 - aXZ^2 - bZ^3
    \end{equation*}
\end{itemize}
\end{frame}

%------------------------------------------------

\begin{frame}
\frametitle{\insertsection~--~\insertsubsection}
\begin{itemize}[<+->]
	\item Wir betrachten die elliptische Kurve
		\begin{equation*}
			E=\{(x:y:z)\mid 0=y^2z - x^3 - axz^2 - bz^3\}
		\end{equation*}
	\item Wir erinnern uns an die unendlich fernen Gerade $U=\overline{(0:1:0),(1:0:0)}$.
	\item Es gilt: $U\cap E=(0:1:0)\eqqcolon\patinf$, d.h.\ es liegt nur $\patinf$ auf unserer Kurve $E$.
	\item Wir bezeichnen $\patinf$ als den \textbf{unendlich fernen Punkt}.
\end{itemize}
\end{frame}

%------------------------------------------------

\section{First Section} % Sections can be created in order to organize your presentation into discrete blocks, all sections and subsections are automatically printed in the table of contents as an overview of the talk
%------------------------------------------------

\subsection{Subsection Example} % A subsection can be created just before a set of slides with a common theme to further break down your presentation into chunks

%------------------------------------------------

\begin{frame}
\frametitle{Bullet Points}
\begin{itemize}
\item Lorem ipsum dolor sit amet, consectetur adipiscing elit
\item Aliquam blandit faucibus nisi, sit amet dapibus enim tempus eu
\item Nulla commodo, erat quis gravida posuere, elit lacus lobortis est, quis porttitor odio mauris at libero
\item Nam cursus est eget velit posuere pellentesque
\item Vestibulum faucibus velit a augue condimentum quis convallis nulla gravida
\end{itemize}
\end{frame}

%------------------------------------------------

\begin{frame}
\frametitle{Blocks of Highlighted Text}
\begin{block}{Block 1}
Lorem ipsum dolor sit amet, consectetur adipiscing elit. Integer lectus nisl, ultricies in feugiat rutrum, porttitor sit amet augue. Aliquam ut tortor mauris. Sed volutpat ante purus, quis accumsan dolor.
\end{block}

\begin{block}{Block 2}
Pellentesque sed tellus purus. Class aptent taciti sociosqu ad litora torquent per conubia nostra, per inceptos himenaeos. Vestibulum quis magna at risus dictum tempor eu vitae velit.
\end{block}

\begin{block}{Block 3}
Suspendisse tincidunt sagittis gravida. Curabitur condimentum, enim sed venenatis rutrum, ipsum neque consectetur orci, sed blandit justo nisi ac lacus.
\end{block}
\end{frame}

%------------------------------------------------

\begin{frame}
\frametitle{Multiple Columns}
\begin{columns}[c] % The "c" option specifies centered vertical alignment while the "t" option is used for top vertical alignment

\column{.45\textwidth} % Left column and width
\textbf{Heading}
\begin{enumerate}
\item Statement
\item Explanation
\item Example
\end{enumerate}

\column{.5\textwidth} % Right column and width
Lorem ipsum dolor sit amet, consectetur adipiscing elit. Integer lectus nisl, ultricies in feugiat rutrum, porttitor sit amet augue. Aliquam ut tortor mauris. Sed volutpat ante purus, quis accumsan dolor.

\end{columns}
\end{frame}

%------------------------------------------------
\section{Second Section}
%------------------------------------------------

\begin{frame}
\frametitle{Table}
\begin{table}
\begin{tabular}{l l l}
\toprule
\textbf{Treatments} & \textbf{Response 1} & \textbf{Response 2}\\
\midrule
Treatment 1 & 0.0003262 & 0.562 \\
Treatment 2 & 0.0015681 & 0.910 \\
Treatment 3 & 0.0009271 & 0.296 \\
\bottomrule
\end{tabular}
\caption{Table caption}
\end{table}
\end{frame}

%------------------------------------------------

\begin{frame}
\frametitle{Theorem}
\begin{thm}[Mass--energy equivalence]
$E = mc^2$
\end{thm}
\end{frame}

%------------------------------------------------

\begin{frame}[fragile] % Need to use the fragile option when verbatim is used in the slide
\frametitle{Verbatim}
\begin{bsp}[Theorem Slide Code]
\begin{verbatim}
\begin{frame}
\frametitle{Theorem}
\begin{theorem}[Mass--energy equivalence]
$E = mc^2$
\end{theorem}
\end{frame}\end{verbatim}
\end{bsp}
\end{frame}

%------------------------------------------------

\begin{frame}
\frametitle{Figure}
Uncomment the code on this slide to include your own image from the same directory as the template TeX file.
%\begin{figure}
%\includegraphics[width=0.8\linewidth]{test}
%\end{figure}
\end{frame}

%------------------------------------------------

\begin{frame}
\frametitle{References}
\footnotesize{
\begin{thebibliography}{99} % Beamer does not support BibTeX so references must be inserted manually as below
\bibitem[Smith, 2012]{p1} John Smith (2012)
\newblock Title of the publication
\newblock \emph{Journal Name} 12(3), 45 -- 678.
\end{thebibliography}
}
\end{frame}

%------------------------------------------------

\begin{frame}
\Huge{\centerline{The End}}
\end{frame}
%----------------------------------------------------------------------------------------
\end{document} 
